\documentclass[a4paper,11pt]{article}
\usepackage[plain]{fullpage}
\usepackage{graphicx}  %This enables the inclusion of pdf graphic files in figures
\usepackage{wrapfig}
%\addtocounter{section}{0}
\title{Lab Project,\\Final report}
\author{Group 5 \\Magnus Krane, Erik S\o rensen, Piyush Bajpayee}
\date{ {\tt \{eriksore, magnkr, piyushb\}@stud.ntnu.no}\\
TTM4135 Information Security\\
\today}
\begin{document}
\maketitle
\vspace{3cm}
% Abstract
\begin{abstract}
This lab report is a term assignment in the course TTM4135 at NTNU. 
\end{abstract}
\section{Introduction}
\paragraph{}The web could be argued to be one of the most important infrastructures in the world.
\section{Experimental procedure}
\paragraph{}During the experiment we followed \cite{1} point by point, therefore only questions asked in the text (other than Q-questions) and points were we changed procedure will be mentioned in this part.
\paragraph{Part 1.4}The purpose of the 'echo' and 'touch' commands were to create a "database" for the certificates.\\
In the \emph{caconf.cnf} file the default-md and policy-match variables were changed. default-md were set to SHA1, which is more secure than md5 \cite{3}, and more matches were set in policy-match. These steps were done to make the certificate more secure.
\section{Results}
\section{Discussion}
\subsection{Q1}
\paragraph{Comment on security related issues regarding the cryptographic algorithms used to
generate and sign your groups web server certificate (key length, algorithm, etc.).}
\paragraph{} For the web server certificate, the signature algorithm were chosen as SHA1 with RSA encryption, with a public-key length of 2048 bits. At the current time, RSA laboratories (creators of the RSA algorithm) recommends a 2048 bits key size for extremely valuable keys, and 1024 bits key size for corporate use \cite{2}. Even 768 bits key might be regarded as sufficient. The use of 2048 bits key size in this case could therefore be regarded as 'over-the-top'. Ultimately the choice of key length is a trade-off between de-/encryption speed (see \cite{5} page 301) and security.\\
SHA1 (160-bit hash) were chosen over Md5 (128-bit hash) as message digest due to known security problems with Md5 \cite{3}
\subsection{Q2}
The detached PGP signature of the source code should be verified using GnuPG.
\paragraph{Explain what you have achieved through each of these verifications. What is the name of the person signing the Apache release?}
\paragraph{}Through these step we have identified that a "Jim Jagielski" has signed the file, but we can still not trust the file as it is not certified with a trusted signature. A check against the pgpkeys.mit.edu keyserver also shows this. In conclusion, the received public key can't be trusted. To validate the authenticity of a key (and therefore also, the integrity of the file) \cite{4} recomends a face-to-face validation with the signee. 
\subsection{Q3}
\paragraph{What are the access permissions to your web servers configuration files, server certificate and the corresponding private key? Comment on possible attacks to your web server due to inappropriate file permissions.}
\paragraph{}The following file permissions have been set.
\begin{itemize}
	\item Configuration files: 600, -rw- --- ---
	\item Server certificate: 400, -r-- --- ---
	\item Private key: 400, -r-- --- ---
\end{itemize}
\paragraph{} These permissions has been set as restrictive as possible. Especially for the certicate and key files, their persmissions have been set to read only. This is to make sure their integrity is kept. All files have been given read/write access for owner only, i.e. you would need to be user root/gr05 to read these files. The files is also stored in folders not accesible from the Internet.
\paragraph{}Possible attacks could be to change certificate requirements in the configuration file to gain access, or if the private key were to be accessed the whole certificate access control would be rendered useless.
\subsection{Q4}
\paragraph{Web servers offering weak cryptography are subject to several attacks. What kind of attacks are feasible? How did you configure your server to prevent such attacks?}
\begin{itemize}
	\item \textbf{Brute force attacks} are a type of attacks were all possibilites are tried (in this case for the private key). This attack will not be feasible on our server as a key space of 2048 is regarded as safe with todays computing power and factoring algorithms \cite{5}.
	\item \textbf{Encrypted passwords in database} 
	\item \textbf{Password exhaustive/brute force attacks:} The cryptography protecting the server is worth nothing should an attacker gain access to the root user. This could be done through a brute force/dictionary attack against the servers password (see \cite{6} page  276 for details). If such a attack were successfull, an attacker would have full access to the whole server and database. Creating a strong password will be the best defence against such an attack.
\end{itemize}
\subsection{Q5}
\paragraph{What kind of malicious attacks is your web application (PHP) vulnerable to? Describe
them briefly, and point out what countermeasures you have developed in your code to prevent
such attacks.}
\paragraph{}Out application is vulnerable to Cross-Site Scripting (XSS), Cross-Site Request Forgery (CSRF), Session Management, and SQL-Injection. XSS and CSRF uses input fields on a site to upload html that includes sripts, like javascript that the users then reads and executes (\cite{8} page 734).
\paragraph{}We are also vulnerable to SQL-Injection. This is when the developer passes user input to a back-end database without checking to see whether it contains SQL-code. This is often given away by error messages, from which the evil doer may interpret to mount a attack (\cite{8} page 120). Session Management could executed when the evil doers have gotten a session from another user and they edit the trafic between them and the server to act as the user.
\paragraph{}Since none of us had any experience with PHP, we did not manage to make it as secure as we would like. If we look at the most critical web application security flaws \cite{7}, we can see that we didnt acount for many of them. 
\section*{Conclusion}
\begin{thebibliography}{9}
\bibitem{1}ITEM, 
	\emph{Part 3 - Design and implementation of a resource allocation service}.
	NTNU, 2013. \\http://www.item.ntnu.no/fag/ttm4120/current/lab/lab3spread.pdf, retrieved Feb 27th 2013.
\bibitem{2}RSA Laboratories, \emph{How large a key should be used in the RSA cryptosystem?}.\\ http://www.rsa.com/rsalabs/node.asp?id=2218, retrieved Mar 8th 2013.	
\bibitem{3}IETF.org, \emph{Updated Security Considerations for the MD5 Message-Digest and the HMAC-MD5 Algorithms}.\\ http://tools.ietf.org/html/rfc6151, downloaded Mar 9th 2013.
\bibitem{4} Apache.org, \emph{Verifying Apache HTTP Server Releases}.\\ http://httpd.apache.org/dev/verification.html, retrieved Mar 10th 2013.
\bibitem{5}William Stallings, \emph{Cryptography and Network Security - Principle and Practice}. Fifth edition, Prentice Hall, 2011.
\bibitem{6}Charles P. Pfleeger, \emph{Security in Computing}. Fourth edition,Prentice Hall, 2006.
\bibitem{7}Owasp.org \emph{Owasp Top Ten Project}.\\ https:www.owasp.org/index.php/Category:OWASP\textunderscore Top\textunderscore Ten \textunderscore Project
\bibitem{8}Ross Anderson \emph{Security Engineering. Second edition}, Wiley Publishing, 2008.
\end{thebibliography}
\listoffigures
\end{document} 
